\documentclass[final]{elsarticle}
%% \documentclass[final,times,twocolumn]{elsarticle}
\usepackage{lineno,hyperref}
\modulolinenumbers[5]
\journal{Journal of \LaTeX\ Templates}

%===========----------------------Package----------------------===========

\usepackage{lineno,hyperref}
\modulolinenumbers[5]
\usepackage{graphicx}
%\usepackage{cite}
\usepackage{amsmath,amssymb,amsfonts}
\usepackage{float}
\usepackage{graphicx}
\usepackage[justification=centering, bf]{caption}
\usepackage{textcomp}
\usepackage[ruled, resetcount, linesnumbered]{algorithm2e}
\usepackage{array}
\usepackage{booktabs}
\usepackage{multirow}
\usepackage{color}
%\usepackage{ulem}
\usepackage{rotating}
\usepackage{pdflscape}
\usepackage{array, booktabs, tabularx} 
\usepackage{setspace}
\usepackage{booktabs}
\usepackage[table,xcdraw]{xcolor}
\usepackage{rotating}
\usepackage{longtable}
\usepackage{verbatim}
\modulolinenumbers[5]
\journal{Journal of \LaTeX\ Templates}
\newcommand{\vect}[1]{\overrightarrow{\boldsymbol{#1}}}
\newcommand{\uvec}[1]{\boldsymbol{\hat{\textbf{#1}}}}
%%%%%%%%%%%%%%%%%%%%%%%
%% Elsevier bibliography styles
%%%%%%%%%%%%%%%%%%%%%%%
%% To change the style, put a % in front of the second line of the current style and
%% remove the % from the second line of the style you would like to use.
%%%%%%%%%%%%%%%%%%%%%%%

%% Numbered
%\bibliographystyle{model1-num-names}

%% Numbered without titles
%\bibliographystyle{model1a-num-names}

%% Harvard
%\bibliographystyle{model2-names.bst}\biboptions{authoryear}

%% Vancouver numbered
%\usepackage{numcompress}\bibliographystyle{model3-num-names}

%% Vancouver name/year
%\usepackage{numcompress}\bibliographystyle{model4-names}\biboptions{authoryear}

%% APA style
%\bibliographystyle{model5-names}\biboptions{authoryear}

%% AMA style
%\usepackage{numcompress}\bibliographystyle{model6-num-names}

%% `Elsevier LaTeX' style
\bibliographystyle{elsarticle-num}
%%%%%%%%%%%%%%%%%%%%%%%

\begin{document}
\begin{frontmatter}
\title{A Family System based Genetic Algorithm for  Obstacles-Avoidance Minimal Exposure Path Problem in Wireless Sensor Networks}

%%% Group authors per affiliation:
%\author[httb]{Huynh Thi Thanh Binh}
%\ead{binhht@soict.hust.edu.vn}
%
%%% or include affiliations in footnotes:
%\author[ntmb]{Nguyen Thi My Binh\corref{cor1}}
%\ead{binhdungminhkhue@gmail.com}
%\author[httb]{Nguyen Hong Ngoc}
%\ead{ngocnguyen.nd97@gmail.com}
%\author[dthl]{Dinh Thi Ha Ly}
%\ead{greeny255@gmail.com }
%\author[httb]{Nguyen Duc Nghia}
%\ead{nghiand@soict.hust.edu.vn }
%\cortext[cor1]{Corresponding author. Tel: +84 977901599}
%\address[httb]{ Hanoi University of Science and Technology, Vietnam}
%\address[ntmb]{Hanoi University of Industry, Vietnam}
%\address[dthl]{National Institute of Informatics, Tokyo, Japan}
%\fntext[myfootnote]{Since 1880.}
%% or include affiliations in footnotes:
%\author[mymainaddress,mysecondaryaddress]{Nguyen Thi My Binh}
%\ead[url]{www.elsevier.com}
%
%\author[mysecondaryaddress]{Hanoi University of Science and Technology \corref{mycorrespondingauthor}}
%\cortext[mycorrespondingauthor]{Huynh Thi Thanh Binh}
%\ead{support@elsevier.com}
%
%\address[mymainaddress]{Hanoi University of Science and Technology, Vietnam}
%\address[mysecondaryaddress]{360 Park Avenue South, New York}
%\address[mysecondaryaddress]{360 Park Avenue South, New York}
\begin{abstract}
Barrier coverage in wireless sensor networks has received drawing attention to the research community in recent years, due to its advantages for security applications. One of fundamental barrier coverage problems is minimal exposure path (MEP), which is the problem of finding the path across the sensor network with the lowest chance of being detected. This problem plays an important role in the applications for detecting intrusion and evaluating the effectiveness of coverage in a sensing field. However, in the majority of researches, sensors studied network were omni-directional and/or homogeneous. In contrast, this paper investigates the minimal exposure path problem in heterogeneous directional wireless sensor networks (hereinafter O-based-MEP). The O-based-MEP is more practical, meaningful and complex with the unique characteristics of directional sensor nodes. The O-based-MEP is converted into a numerically function extreme which is high dimensional, non-differentiable and non-linear. Adapting to these features, we propose two efficient meta-heuristic algorithms, HDWSN-EA and HDWSN-PSO, to solve the converted problem. HDWSN-EA is formed by the evolution algorithm with a featured individual representation and an effective combination of evolution operators while HDWSN-PSO is an improved on the characteristics of a particle swarm population. Experimental results on numerous instances indicate that the proposed algorithms are suitable for the converted O-based-MEP problem and perform well regarding both solution accuracy and computation time compared to existing approaches.
  
\end{abstract}
\begin{keyword}
\texttt{Minimal exposure path} \sep \texttt{Directional sensing coverage model} \sep \texttt{Heterogeneous directional wireless sensor network} \sep \texttt{Evolution algorithm} \sep\texttt{Particle swarm optimization algorithm}
%\texttt{elsarticle.cls}\sep \LaTeX\sep Elsevier \sep template
%\MSC[2010] 00-01\sep  99-00
\end{keyword}
\end{frontmatter}
%\linenumbers
\section{Introduction}
Wireless Sensor Networks (WSNs) have drawn much attention recently because of their vast potential applications in various areas from commercials to law enforcement, from civil to military. The main application of WSNs is to monitor the sensing field and detect intrusion. The intruding object is detected when it crosses a border or penetrates a protected area. This is often referred to as coverage of intrusion path problem. This declaration, in contrast with the full coverage problem which requires coverage on every point of interested region, only consider detecting a certain intruding objects when it penetrates the sensing area. The coverage can be considered as a measure of surveillance quality as well as service efficiency that WSNs can provide. Based on coverage computation, weak points in a sensor field can be discovered, guiding to redeploy the network in the future and reconfigures schemes to improve the overall service efficiency as well as surveillance quality.

Exposure, which shows how well an object traversing in an arbitrary path through the sensor network can be observed by the sensor network over a period of time, is directly related to coverage. Therefore, exposure is used as a good measure for assessing the coverage quality of the sensor network. The higher the exposure, the more consistently the sensor network can detect the intruders. The exposure of a penetration path in a sensing field is equivalent to the ability to detect a target traveling along that path. The path along which the ability to be detected is minimized, called the minimal exposure path and the problem that looks for such a path is Minimal Exposure Path (MEP) problem. The MEP offers valuable information about bad coverage paths in sensor networks. An object moving through a sensor field along the MEP is often the most difficult to be detected. The analysis of the MEP permits one to anticipate the weak points of a sensor network and propose a strategy to improve it if necessary. The MEP is not only useful to assess the service quality of WSNs, but it is also a measure of deployment effectiveness. The information of exposure can be used in managing, optimizing and maintaining the WSNs.

WSNs are classified based on type of sensors such as temperature, infrared, humidity and video sensors. Different sensor types require different sensing models. A type of sensing model should be able to describe the sensitivity or the capability of the sensor \cite{b10}. Sensing model can be classified into two subcategories: mathematical and physical. The mathematical sensing model, which includes binary and probabilistic models, expresses the sensitivity of sensors; the physical sensing model provides insights into the sensing direction of the sensor node. There are two different physical sensing models: omni-directional sensing and directional sensing model. Many existing sensor node types have omni-directional model such as magnetic, temperature and humidity can sense in all direction, such model is a traditional sensing model. In contrast, a directional sensor cannot sense in all 360 degrees. A sensor network is called a directional wireless sensor network (DWSN) if all the sensors comply with directional sensing model. With advancement in multimedia technology, there are many practical applications of wireless multimedia sensor networks, in which sensors usually have a directional sensing model known as DWSNs. DWSNs have distinctive characteristics of each type. For example, camera sensors, radio sensor, ultrasound sensors, radar sensors and infrared sensors have different properties. Furthermore, DWSNs can retrieve higher levels of sensor information or richer information form such as audio, image and video, thus providing more detailed information of environment. Therefore, DWSNs have been grown in popularity, attracted increasing interest and gained much importance in recent years. However, they use a larger amount of the limited energy reserve resource available to a wireless sensor node. Also, coverage holes problem will arise in the case of the same number of directional sensors deployed in a given region. These features cause the majority of existing coverage control theories and methods to be not directly applicable to DWSNs. 

Besides, in a typical target tracking scenario, moving objects tend to occur randomly and are often associated with various detectable physical signals.
So it requires a heterogeneous directional wireless sensor networks (HDWSN). A HDWSN is a sub-class of DWSNs in which each directional sensor node may possess different computational capabilities, different energy storage devices, a different number of sensing units, and use different communication links. These characteristics make the HDWSN a very complex-to-study network, however, the real motivation behind the HDWSNs is the need to conserve energy and complex hardware embedded without reduce the overall sensing effectiveness, hence this may decrease the total cost of operating the sensor network. Therefore, specific solutions and techniques are required with a view to enhancing network efficiency, evaluating coverage quality, and increasing the performance of HDWSNs. 

Unlike most existing works about the MEP problem in traditional WSN, which are based on either omnidirectional sensors and/or homogeneous network, this paper focuses on solving the MEP in HDWSNs problem, which is more valuable in practical applications. To the best of our knowledge, we are the first to investigate the MEP problem in HDWSN.

The main contributions of this paper are as follows:
\begin{itemize}
	\itemsep0em
	\item Formulate the MEP problem in HDWSNs known as O-based-MEP.
	\item Convert the O-based-MEP into an optimization model with constraints, which allows mathematical optimization methods to be used in tackling the problem. 
	\item Propose two efficient meta-heuristic algorithms, namely HDWSN-EA and HDWSN-PSO to solve the O-based-MEP. HDWSN-EA is a hybrid of evolution algorithm and local search. HDWSN-PSO is an improved particle swarm optimization, which can find out the approximate optimal solution in shorter time.
	\item Conduct comprehensive experiments to test the efficiency of the proposed algorithms.
	\item Compare our meta-heuristic algorithms with the state of the art approximate algorithms which could achieve the current best approximation ratio and the best meta-heuristic approximation.
	\item Provide an analysis of the obtained results and compare with the results by existing methods. Experimental results show that our proposed algorithms outperform the classical algorithms for most cases regarding solution accuracy and computation time.	
\end{itemize}

The rest of the paper is organized as follows. Related works are presented in Section 2. Preliminaries and formulation for the O-based-MEP problem are discussed in Section 3. Section 4 introduces the proposed algorithms. Experiments results examining the proposed algorithms along with computational and comparative results are given and analyzed in Section 5. Finally, Section 6 presents conclusions and future works of the paper.

\section{Related Works}
The MEP problem in WSN depends on the approach method, which is consisted of various factors such as type of sensors, deployment strategy, problem modelling, tackling algorithm, etc. Therefore, this part briefly represents related works to O-based-MEP problem and analyses some advantages and drawbacks of each approach to the Minimal Exposure Path problem.

Since Meguerdichian et al. claimed the Minimal Exposure Path problem in [1] in 2001, there have been numerous studies regarding this problem in various conditions. The article also defines the exposure of a point in a sensing field, which is a value calculated through the exposure of sensors in that field corresponding to the considered point, in 2 important models, which is the All-Sensor Field Intensity and the Closest-Sensor Field Intensity. While the first model shows the exposure as the sum of the exposure of all sensors in the field corresponding to the considered point, the second model only accounts the exposure of the nearest sensor, which is the one with highest exposure supervising the point, in the final value. While the latter model is fairly easier to find the solution, it is not as effective to present the problem in reality as the former one due to the ignorance of the cooperation of several sensors to detect an intruder, and in general, the total exposure to the considered point from the other sensors is usually larger than that from only the maximum one. Meguerdichian et al. also successfully found the precise solution to the Minimal Exposure Path problem for the field with only one sensor, which offers valuable and profound understandings in the Minimal Exposure Path problem itself.

Further processes were achieved in [2] when Clouqueur et al. published the probability model to describe the Minimal Exposure Path problem as the problem of finding the path with the lowest chance of detection. In the article, the chance of detection is calculated as the probability in discrete point along the path of the moving intruder. Furthermore, the article also presents a grid-based method to find the solution to the Minimal Exposure Path problem. However, the approach shows significant mistakes. The most considerable is resulted from their misunderstanding about the concept of detection probability. Specifically, the lacking of time factor in the model cause a principal hole in the logical process from the discrete value (the detection probability at a certain point) to the continuous measure (the detection chance along the path which contains an infinite amount of distinguished points). As a result, their model is constrained with their grid arrangement method and inefficient to approach a real Minimal Exposure Path problem.

The directional sensor model is first presented in [3] by Ma H. and Liu Y. The directional sensors differs from the omnidirectional ones by their different sensing values in different directions when the distances are the same. The directional sensors are described through several parameters in addition to the parameters required in an omnidirectional ones. Then, the directional sensor model is considered in the Minimal Exposure Path problem by Adriaens et al. in [4]. The articles shows a method of finding the Maximal Breach Path, which can be considered as a model for the Minimal Exposure Path problem, by using the Voronoi graph method. Despite being the first one studying the Minimal Exposure Path problem, the article uses a simple model to present the problem, which is not efficient to accurately model the real problem.

In [5], Skraba and Guibas introduces the existence of obstacles inside the sensing field which can block not only the moving path but also the sensing signal of sensors. As a result, one sensor can only sense an intruder if it is inside the sensor's field of view and no obstacle appear in the straight line between the intruder and the sensor. Furthermore, the article gives a clear definition of the detection probability with the consideration of time, which clarifies the logical hole in the probability model in [2]. However, their possibility of detection comes from the chance that certain sensors are off, which means that the model is actually a binary coverage model and the probability appears only because of the lack of information about the states of the sensors.

In [6], Wu and Chung introduce the heterogeneous wireless network sensor model and illustrate its exceptional advantages compared to the homogeneous one. The first upside of the heterogeneous model is the flexibility of the sensor network to adapt several environmental factors while the second advantage offers the generality which can precisely model the problem in reality. The article also analyses the detection probability correlate to the distance from the sensor and effectively change the problem from a probability perspective to a binary one. However, due to the absence of time factor, their model is more effective to analyse exposure on steady points rather than a moving object through the sensing field.

In [7], Liu et al. offer an approximation algorithm to solve the Minimal Exposure Path problem regarding sensitivity model, where the exposure of a point to a sensor is evaluated taking into account the direction of the sensor and the distance between it and the considered point. They propose algorithms to solve both the All-Sensor Field Intensity and the  Maximum-Sensor Field Intensity by using a grid-based model and Djikstra's shortest path algorithm. However, since they force the object moving along the grid, the solution may be potentially inaccurate due to the limitation in moving direction compared to the situation in reality.

Ye and Wang later introduce a more precise approach to the Minimal Exposure Path problem in [8]. They model a path by a set of points which has a same displacement in abscissa. With the assumption of maximum speed moving, the exposure along a small part is computed as a product of the length of each part and the exposure value at the former point on that part. The genetic algorithm is then conducted, with several optimisation to prevent parts with large length, which can cause the population trapping in a local optima and a low convergence speed. However, this model still has its own disadvantages. Firstly, the path is strict to moving forward without being able to make a turn around, which may ignore some valuable path which turn around to get away from sensors. Secondly, the fixed displacement of abscissa leads to a limit displacement of ordinate, which limit the angle at which the object can move, further reduce the possibility of finding the most preferable solution.

In [9], Liu et al. give a clear definition to the Minimal Exposure Path problem with heterogeneous sensors and obstacle, the intruder has to move through the sensing field not going across existed obstacles and avoiding being detected by several sensor types. The path is presented through a system of adaptive ordered grids and the solution may converge to the optimal solution when the adaptability and order of the grid system approach infinity. However, there are still significant drawbacks. Firstly, the grid-based method limits the accuracy of the final solution, and the effort of fixing the issue using adaptive ordered cells may lead to a dramatic surge in required calculation time without achieving considerably more preferable results. Secondly, the simulation results are based on only one topology of obstacles, which may lead to untrusted results due to low-size data.

\section{Preliminaries and Problem Formulation}

In this section, different subjects of the problem will be examined and transformed into mathematical model. The subsection Preliminaries will define the model of sensor and obstacle as well as propose a mathematical equation for the minimal exposure path. After that, the problem will be formulated under a set of input parameters and output value in the subsection Problem Formulation.

\subsection{Preliminaries}

\subsubsection{Sensor model}

Based on the specific requirements, there are many kinds of information that a sensor can sense e.g. temperature, humidity, infrared, and video etc. Sensors can also be categorized based on sensing model, which is a mathematical definition that expresses the sensitivity or the capability of the sensor. Different types of sensing model can be perceived through many related researches in this field.

\textbf{Coverage Model}

The omni-directional sensor is the most basic type of coverage model, by that, the sensor's sensing region is in the shape of a disk. An omni-directional sensor comes with a parameter called the sensing radius $r$, which stands for the radius of the sensing region. An object $ O $ is said to be covered by a omni-directional sensor $ s $ if and only if the Euclidean distance between the position of sensor $s$ and object $ O $ less than or equal the sensing radius $r$.

A higher transformation of the omni-directional coverage model is the directional coverage model, by that, a sensor is only capable of sensing in a bounded direction. The sensors of this type often can be found as security cameras in realistic scenarios. For a mathematical definition, in 2\_D dimension, the sensing area of a directional sensor $ s $ is a sector denoted by 4-tuple $( P, r, \overrightarrow{Wd}, \alpha )$. Where $ P $ denotes the location of the sensor, $ r $ the sensing radius, $ \overrightarrow{Wd}$ the unit vector showing working direction and $ \alpha $ the angle of view. The directional sensing capability is illustrated in Fig. \ref{Fig.1}. \\
\begin{figure*}[h]
	% Use the relevant command to insert your figure file.
	% For example, with the graphicx package use
	\centering
	\includegraphics[width=0.4\textwidth]{mptt/sensorView}
	% figure caption is below the figure
	\caption{Sensing capability of directional sensor}
	\label{Fig.1}       % Give a unique label
\end{figure*}
An object $ O $ is said to be covered by a directional sensor $ s $ if and only if the following conditions are met: 
\begin{itemize}
	\itemsep0em
	\item $d(P,O) \le r$ or $\left\| {\overrightarrow {PO} } \right\| \le r$, where $d(P,O)$ is the Euclidean distance between the position $ P $ of sensor $s$ and object $ O $.
	\item The angle between $\overrightarrow{PO}$ and $\overrightarrow {Wd} $ is within $\left[ { - \frac{\alpha}{2} ,\frac{\alpha}{2} } \right]$ or $\overrightarrow {PO} .\overrightarrow {Wd}  \ge \left\| {\overrightarrow {PO} } \right\|\cos \frac{\alpha}{2} $	
\end{itemize}

\textbf{Sensing Intensity Model}

Sensing intensity of a sensor on a target point is a value denotes how well the sensor to be able to sense the target. The model of sensing intensity often comes in the form of a function of the sensor and the target point. 

The most simple model is the Boolean Sensing Intensity function, which defines the sensing intensity of a sensor $ s $ on a target point $ O $ is $1$ if $O$ is inside the sensing region of $s$ and is $0$ otherwise. A more realistic version of this model is the Attenuated Sensing Intensity model, which said, the closer to the target the stronger the sensing intensity of the sensor. In the omni-directional coverage model, the sensing intensity function is inversely proportional with the Euclidean distance between the sensor and the target. In the directional coverage model, the sensing intensity of a sensor $s$ on a target $O$ is also inversely proportional with the offset angle between the working direction $\overrightarrow{Wd}$ and vector $\overrightarrow{PO}$. The sensing intensity functions of these two models are shown in following:

\textit{Attenuated Sensing Intensity Model for Omni-directional Sensor}\\
\begin{equation}
\label{eqfo}
f_o({s_i},O) = \frac{C}{{{{\left[ {d(P,O)} \right]}^\lambda }}}
\end{equation}

\textit{Attenuated Sensing Intensity Model for Directional Sensor}\\
\begin{equation}
\label{eqfd}
f_d({s_i},O) = \frac{{C{{\left\{ {\cos \left( {\frac{{\angle (\overrightarrow {PO} ,\overrightarrow {Wd}) }}{2}} \right)} \right\}}^\beta }}}{{{{\left[ {d(P,O)} \right]}^\lambda }}}
\end{equation}

Where $ d(P, O) $ is the distance between $ P $ and $ O $; $ \angle (\overrightarrow {PO}; \overrightarrow {Wd})$ is the angle between $ \overrightarrow {PO} $ and $ \overrightarrow {Wd}$ , $C$ is a constant; $ \lambda,\ \beta $ are the sensibility attenuation exponents. 

\begin{figure*}[htbp!]
	% Use the relevant command to insert your figure file.
	% For example, with the graphicx package use
	\begin{tabular}{cc}
		\includegraphics[width=0.3\linewidth]{epsfile1/b1y1}&\includegraphics[width=0.3\linewidth]{epsfile1/b1y2}\\
		(a) $\beta =1, \lambda=1 $ &(b)$ \beta=1, \lambda=2 $\\
		\includegraphics[width=0.3\linewidth]{epsfile1/b2y2}&\includegraphics[width=0.3\linewidth]{epsfile1/b4y1}\\
		(c) $ \beta=2, \lambda=2 $& (d)$ \beta=4, \lambda=4 $\\
	\end{tabular}
	% figure caption is below the figure
	\centering
	\caption{Illustration attenuated directional sensing model with different $ \beta's $ and $ \lambda's $
	}
	\label{Fig.2}       % Give a unique label
\end{figure*}

\textbf{Summary of Sensing Intensity}

%An object at point $ O $ may be detected by several directional sensors. In Boolean directional sensing model, the coverage of that object by all sensors in the given field $\Omega$ can be measured by accumulation of the coverage of all sensors. Thus, the sensing intensity function is defined under the following model.
%\begin{equation}
%\label{eqib}
%I_b(O) = \sum\limits_{i = 1}^N {{f_b}} ({s_i},O)
%\end{equation}
%where$ N $ is number of sensors in field $\Omega$ . 
%
%\textbf{\textit{Attenuated Directional  Sensing Model}}

%Even though sensors commonly have widely different theoretical and physical characteristics, most types of sensors share the following property: the closer the object, the more likely the sensor can detect or cover it. In other words, the sensitivity gradually attenuates with increasing distance. For directional sensors, the sensitivity also attenuates with increasing offset angle from the sensor direction, see Fig.\ref{Fig.2}. In the following definition, the attenuated directional sensing model is interpreted which describes the relationships among sensitivity, the distance, and the offset angle.
%\begin{figure*}[h]
%	% Use the relevant command to insert your figure file.
%	% For example, with the graphicx package use
%	\centering
%	\includegraphics[width=0.25\textwidth]{epsfile1/b1y1 &	\includegraphics[width=0.25\textwidth]{epsfile1/b1y2
%		&	\includegraphics[width=0.25\textwidth]{epsfile1/b2y2
%		&	\includegraphics[width=0.25\textwidth]{epsfile1/b4y4}
%	% figure caption is below the figure
%	\caption{Illustration of the attenuated sensing model }
%	\label{Fig.2}       % Give a unique label
%\end{figure*}

%As the sensing quality of a sensor decreases with the increase of distance away from the sensor, for a given directional sensor $ s $, the coverage only needs to be characterized by a 2-tuple $ (P, \overrightarrow{Wd}) $. The attenuated sensing function in the directional model of directional sensor $ s $ at position $ P $ sensing object at point $O$ is given by:


In the case there are multiple sensors sensing on a target point, summary of sensing intensity is used. Under the Attenuated Sensing model, the coverage of an object $ O $ by all sensors in the field can be calculated by the following Accumulative Intensity function: the sensing intensity on a given target point $ O $ of $N_s$ sensors is defined as sensitivity accumulation of all sensors on $ O $. We denoted the accumulative intensity function by $ I_{sum} $.
\begin{equation}
\label{eqia}
I_{sum}(O) = \sum\limits_{i = 1}^{N_s} {f({s_i},O)} 
\end{equation}
where $ f $ is sensing intensity function of a sensor given above and $ N_s $ is the number of sensors in the field.

%\textit{Closest-sensing intensity function:} the sensing intensity on a given target point $ O $ is defined as the intensity measured by the closest sensor, i.e, the sensor which has the smallest Euclidean distance from that object. We denoted closest-sensing intensity function by $ I_c $. The distance and the closest-sensing intensity function are calculated by the following equation. 
%\begin{equation}
%\label {eq6}
%{s_{\min}(O)} = \{ {s_j} \in S\left| {d({s_j},O) \le d({s_i},O) \ \forall {s_i} \in S,i = \overline {1..N} } \right.\} 
%\end{equation}
%\begin{equation}
%\label{eqic}
%I_c(O) = f_a({s_{\min }(O)},O)
%\end{equation}
\subsubsection{Obstacle Model}
In realistic deployment, the region of interest is often filled with multiple obstacles such as lakes, rivers, houses and trees. There are many kinds of obstacle in practical environment with different characteristics. In O-based-MEP problem, an obstacle is modeled as a polygon with ability to decrease or interrupt the sensing signal. In addition, the obstacles also restrict the movement of the intruder as well as the deployment of sensors. 

For mathematic definition, obstacles are convex polygons cover the area that block the movement of the intruder, the deployment and also the sensing signal of sensors. In this paper, an obstacle is represented as an ordered set of $2D$ points $ L_O = \{O_1, O_2,\ldots,O_m\}$ and an absorbability parameter $\nu$ where $\nu \in [0,1]$. The obstacle is the polygon $O_1 O_2\ldots O_m $ created by ordered connecting these points and connecting $O_m$ to $O_1$. For modeling purpose, sensors are not allowed to be deployed within the obstacles area. Similarly, the intruder is not allowed to move within the obstacles area. The obstacles are also be able to absorb the sensing signal of sensors depended on the absorbability parameter $\nu$. To be more specific, assume that the line segment between the sensor $s_i$ and the sensing point $ P $ intersects with an obstacle $O$ having the absorbability parameter of $\nu$, the sensing intensity will be reduced to: 
\begin{equation}
\label{eqob}
f'(s_i,O) = f(s_i,O) * (1-\nu)
\end{equation}
\subsubsection{Minimal exposure path}
Exposure is a value that shows the ability of a sensor network in detecting an object traversing through the sensing field. In O-MEP problem, the exposure value is represented by the path integral of sensing intensity function of sensor field to an unauthorized object along a penetration path. 

On the basis of the aforementioned directional sensing models in Equation \eqref{eqfb}, \eqref{eqfa} and the corresponding sensing intensity functions $I$ i.e, $ I_b $, $ I_a $ or $ I_c $, we further formulate the exposure $ E(I,\wp )$ of a penetration path $ \wp $ from coordinates of the fixed initial position $ B $ on one side and the fixed final position $ E $ on the opposite side of the sensor field $ \Omega $ in directional sensor networks as follows:
\begin{equation}
\label{eqE}
E(I,\wp ) = \int\limits_{\wp }^{} {I(P)} dl
\end{equation}

Equation \eqref{eqE} is non- linear, high-dimensional and non-differentiable. Hence,
it can be solved by partitioning the path $\wp $ into sub-intervals by a set
of points ${L_\wp } = \{ {P_j}\} $ where $j = \overline {0,n} $ and the distance between two arbitrary consecutive points is $\Delta l_j$. Where $\Delta l_j$ is called subinterval and it should be small enough such that the value of function $ {I(P)} $ is similar for each points
lying between those two consecutive points. In Equation \eqref{eqE}, $ E(I,\wp ) $, can be approximately transformed into:
\begin{equation}
\label{eqE1}
E(I,\wp ) \approx \sum\limits_{j = 0}^n {I(P_j)\Delta l_j} 
\end{equation}
By combining Equation \eqref{eqib} and \eqref{eqE1}:
\begin{equation}
\label{eqEb}
E(I_b,\wp ) \approx \sum\limits_{j = 0}^n {{I_b}(P_j)\Delta l_j}  = \sum\limits_{j = 0}^n {\sum\limits_{i = 1}^N {f_b(s_i,P_j)} } \Delta l_j
\end{equation}
By combining Equation \eqref{eqia} and \eqref{eqE1}:
\begin{equation}
\label{eqEa}
E(I,\wp ) \approx \sum\limits_{j = 0}^n {I_a(P_i)\Delta l_i}  = \sum\limits_{j = 0}^n {\sum\limits_{i = 1}^N {f_a({s_i},{P_j})} } \Delta l_j
\end{equation}
By combining Equation \eqref{eqic} and \ref{eqE1}:
\begin{equation}
\label{eqEc}
E(I_c,\wp ) \approx \sum\limits_{j = 0}^n {I_c(P_j)\Delta {l_j}}  = \sum\limits_{j = 0}^n {f_a(s_{\min }(P_j),P_j)} \Delta l_j
\end{equation}
 Later in the experimental results section, the exposure function $ E(I,\wp )$ is going to calculate with different intensity functions $ I_b $,  $ I_a $, $ I_c$ in Equation \eqref{eqEb}, \eqref{eqEa} and \eqref{eqEc} for the purpose of experiment respectively. In which, $ I_b $ will be used for the Boolean model and $ I_a $, $ I_c$ will be used for the Attenuated model.

\subsection{Problem Formulation}
The MEP problem under the assumption of a directional sensing coverage model, O-based-MEP, can be briefly described as follows: Given a set of heterogeneous sensors $S$ of $T$ different types, randomly deployed in the sensor field   and two arbitrary points on opposite sides of the, respectively the source point and the destination point. The goal is to find out a penetration path from the source point $B$ to the destination point $E$ such that an object moves through along path has minimal exposure value. More precisely, the O-based-MEP is formulated as follows.\\
\textbf{Input}
\begin{itemize}
		\itemsep-0.2em
		\item $W$, $H$: width and the length of sensor field $\Omega$
		\item $N$: number of sensors
		\item $ T $: number of sensor types
		\item $ t_i $: number of sensors that belong to type $ i $ ($ i $ = 1, 2,..., $T$), such that:
		 $\sum\limits_i^T {{t_i}}  = N$
		 \item $({x_j},y{}_j)$: position of sensor $ s_j $
		 \item $\overrightarrow{Wd}_j$: working direction of sensor $s_j$
		 \item $ r_i $: sensing radius of type $ i $ ($ i $ = 1, 2, ..., $ T $)
		 \item ${\alpha _i}$: sensing angle of sensor type $ i $ ($ i $ = 1, 2, ..., $ T $).
		 \item $M$: number of obstacles
		 \item $ L_{Oi} $: the set of points of the obstacle $O_i$ ($ i $ = 1, 2,..., $M$)
		 \item $ \nu_i $: the absorbability parameter of the obstacle $O_i$ ($ i $ = 1, 2,..., $M$)
		\item $(0, y_B)$: coordinates of the source point $B$ of the object
		\item $(W, y_E)$: coordinates of the destination point $E$ of the object
\end{itemize}
\textbf{Output:}
\begin{itemize}
	\item A set ${L_\wp }$ of ordered points in $\Omega $ forming a path that connects $ B $ and $ E $ 
\end{itemize}
\textbf{Objective:}\\
The exposure of path  $\wp $ is the smallest, i.e.
\begin{equation}
\label{eqEmin}
{\rm E}(I,\wp ) = \sum\limits_{j = 0}^n {I(P_j)\Delta l_j}  \to Min
\end{equation}
where $ n $ is the number points included in ${L_\wp }$

\textbf{Constraint:}	
\begin{itemize}
%	\itemsep0em	
	\item The object always moves within the sensor field $\Omega $ from $B$ to $E$ with a upper-bounded speed, and can not cross the area of obstacles (*).
	\item The sensors are not deployed inside any obstacles (**)
\end{itemize}
The constraint (*) is to make sure that the distance between any two consecutive sampling points is always the same, which makes it possible to evaluate the average coverage degree of the WSNs.
	
Basically, the general MEP problem is a combinatorial optimization problems. The O-based-MEP has distinctive features, and its the objective function \eqref{eqEmin} is non-linear, high dimensional. To solve efficiently the O-based-MEP problem, we explored a new Evolution Algorithm based on the family system as well as multiple of advanced crossover and mutation techniques. The proposed algorithm is named the Family System based Evolution Algorithm or FEA for short.
%Therefore, after proposing two metaheuristic algorithms, the comparison between performances of the proposed algorithms will hopefully provide insights into advantages of each of the algorithms regarding solution quality and computation time and suggest recommended conditions for applying each algorithm.
\section{Proposed Algorithm}
%Emphasize the difference between standard and family Genetic Algorithm
%Describe the motivation of using GA and the motivation to modify the standard GA as we perform 
An evolutionary algorithm (EA) is a subset of evolutionary computation, a generic population-based metaheuristic optimization algorithm. Furthermore, EA is extremely sufficient for handling noisy functions as well as large and poorly understood search spaces. Due to the chaotic situation of the system of various sensors and obstacles, the O-based-MEP is exceptionally complicated and contains numerous local minima, the EA is the most suitable choice. In addition, EA can also handle large scale and high-dimensional problems well. However, EA as well as most of meta-heuristic algorithms have the disadvantage of having high possibility to be trapped in the local optima. To overcome this, EA is often implemented with a larger size of population to improve the diversity of the population which also results in very high computation time. In the new proposed FEA, we try to lower possibility of local optima without increasing the computation time, by adding the Family System into the population. The reason of EA to be trapped in local optima is the fixed size of population in selection step makes the population less diverse and converge fast. In other words, the selection mays remove many not-yet-good individuals and keep a lot of similar nearly-local-optima individuals. Applying the Family System into EA, our main idea is to create a better crossover and selection method in order to keep the population diverse as much as possible.

In this section, the detail setting and implementation of FEA will be introduced.

%Without loss of generality, we can assume that the mobile object always moves at its greatest speed as the sensors system is stable, the mobile object should move as fast as possible to minimise total time of appearance in the sensor field. Hence, we can calculate the total exposure through the differentiate of not only time but also distance. %insert Appendix above
%
%Consider a path $P$, since it is impossible to calculate the exact value, and acknowledging a proper approximation of the total exposure is enough for application purposes, we will divide the path by several points into small equidistant parts that has approximately equal exposure. As a result, the total value can evaluate through the sum of the exposure in the small parts, which will later be called as $S(P)$. And we can solve the minimal exposure path problem by finding the smallest $S(P)$ possible. Furthermore, since the parts are extremely small compared to the whole path. It is suitable to substitute the distance between two consecutive points on the path by their displacement ($d = \sqrt{dx ^ 2 + dy ^ 2}$)

%In this section, we will introduce a Family-based Genetic Algorithm to effectively tackle this problem.
%
%Due to the chaotic situation of the system of various sensors and obstacles, this problem is exceptionally complicated and contains numerous local minima, the Genetic Algorithm is a suitable choice because the diversity of the population and the stability of complexity may improve the solution over time, avoid being trapped in local minima and get preferable result in acceptable time.

%From experimental result, we can see that (see Experimental Results below), despite its known power and effectiveness, the original Genetic Algorithm seems not to be able to solve the minimal exposure with obstacle problem effectively. It can come from the fact that the individuals of the population can converge comprehensively leads to its falling into a local minima with very low probability of achieving better result.
%
%Therefore, our proposed algorithm will try to find the proper paths while minimising the selection effect on the diversity of the community. Paths are constantly removed despite the total exposure along them, and the selection process only takes place when an over-population occurs. However, the algorithm also constantly stores good paths even after their elimination to archive diverse genes for future generation.

\subsection{Algorithm Modeling}

Before getting further into details about how our proposed algorithm works, it is essential to explain the basic models throughout the performing of the algorithm. Similar to the other EAs, our FEA works with two main model which are the Individual and the Population.

\subsubsection{Individual}

Each individual ($Indi$) stands for a path from the left boundary to the right boundary of the region which satisfy the constraints, or also called a solution (see figure below). An individual is represented with a ordered list of consequent points, starts from the source point and ends with the destination point. To ensure the constraint (*), the distance between two consequent points is not greater than a fixed value of $\Delta s$ and the path formed from ordered connecting these points must become a valid solution. The $n^{th}$ point on $Indi$ is called $I_n$. An individual also has some attracted attributes as following:
%insert figure below
\begin{itemize}
	\item The birth of the individual $Indi.birth$ is the order number of the generation in which the individual is created. The age of the individual will be calculated as the subtraction of the present generation and $Indi.birth$.
%	\item The fitness of the individual $Indi.fit$ is the exposure of the path that the individual represents.
	\item The adult age ($Indi.A$) is the age at which the individual is adult and ready for crossover process.
	\item The death age ($Indi.D$) is the age at which the individual is considered to be dead and removed from the population. 
\end{itemize}
The individual in FEA is different from the individual in standard EA that each has an adult age and a dead age. These two attributes are added in order to control the crossover ability of each individual. By that: 
\begin{itemize}
	\item An individual can only participate in the crossover process after it stays in the population for at least some generations. Being able to stay in the population for some generations proves that the individual is good enough to survive after many selection stages. Furthermore, the individuals which are basically not good from creation will soon be removed after some generations and will not participate in any crossover stages. In summary, the efficiency of crossover stage will be improved and the computation time will be reduced. 
	\item An individual can only participate in the crossover process until its dead after some generations. This constraint makes the individuals which are exist in the population for too long being removed so that the diversity of the population will be improved. This also help reducing the chance of local optima since the local optima individuals will eventually be removed after some generations. Therefore, other individuals which are not-yet-good will have more opportunities to participate in the crossover stage. The genetic resources of these local optima individuals can be preserved by their children and their good attributes will be kept inside the population. However, due to the adult age constraint, these children still can not participate in the crossover process right away but keep staying inside the population.
\end{itemize}  For the purpose of making the modeling simple, every individuals in FEA have the same adult age and death age. The two values will be fixed at the initialization stage of FEA.

\subsubsection{Population}

A population in EA is a set of individuals we perform our algorithm on. It is expected that the lowest fitness of the individuals in the population is constantly decreased each time the algorithm is performed, while the diversity (the quality of a population to contain variety of non-similar individuals) is preserved. The population in FEA named $Pop$ contains all the individuals that are currently alive and participate into the process of the algorithm. Different from the standard EA, in FEA, a new concept of Family is introduced and added to the population system. A Family in the population is defined as a pair of individual and an individual can only perform the crossover process with its own paired individual. An individual in the population can only be in a pair with one and only one other individual or be single. To be more clear, the population $Pop$ is divided into two subsets as following:
\begin{enumerate}
	\item Families set $Family$ is the set of all pairs of individual which make a Family currently exist in the population.
%	and will constantly perform the crossover with its own mate.
%	 The monogamy constraint increase the stability of population genes compared to random crossover situation, which will maximise the probability of getting preferable individuals and minimise the rate of producing terrible ones (details explained below).
	\item Single set $Single$ is the set of all the individuals that are not in any families (or not paired with any individuals). 
%	As a result, they will not perform crossover and their genes are not transmitted to other individual. Taking into consideration that there are also dead individuals in the Singled set, this comes from the fact that the Selection operation does not remove the individual from the Singled set and only marked it as dead.
\end{enumerate}
In addition, some variable are added in order to control the size of the population and the ratio of families in the population, as following:
\begin{itemize}
	\item The number of families in $Family$ is controlled by a static variable called the pairing rate $R_{pair}$ (the minimum ratio of individuals that are in a family).
	\item The population size is controlled by two boundary variables, which are: $p_{min}$ (the minimum number of individuals in $Pop$) and $p_{max}$ (the maximum number of individuals in $Pop$). This is a bit different from the standard EA where the size of population often has only an upper boundary.
\end{itemize}


%Moreover, the population system contains some variables that will help the algorithm acknowledging its flow, in which there are three static variables and one variance variable:
%\begin{enumerate}
%	\item Min population $p_{min}$ is the infimum of the number of individuals in $Pop$. 
%%	The current population reaches its minimum when the population is initialised at the beginning of the algorithm or each time the Selection process is triggered. Moreover, each time after the Selection occurs, if the Singled set has more elements than $p_{min}$, this set is also selected so that only $p_{min}$ individuals left in the Singled.
%	\item Max population $p_{max}$ is the supremum of the population. Each time the current population exceeds this amount, a selection process occurs to remove the individuals with the highest fitness so that the population reduced to the $p_{min}$.
%	\item Pairing rate ($R_{cross}$) is the lower limit of the ratio of the number of individuals which are paired into families. If the current ratio is lower than this infimum, the pairing process will occur.
%	\item Mutation rate ($R_{mutation}$) is the rate of the individuals that are mutated each cycle of the algorithm. The mutation process helps diverse the population in order to avoid local minima and achieve preferable results. 


\subsection{Algorithm Progress}

The proposed FEA is based on the standard EA with the addition of the Family system in order to avoid falling into local optima and improve the efficiency of the standard EA. At the same time, a new crossover operator is also proposed to overcome the challenges and the drawbacks of the previous approaching. The modification focuses on improve the pairing stage, the crossover stage and the selection stage of EA. 

In details, FEA contains six different stages: Initialization, Family Pairing, Crossover, Mutation, Update and Selection. The figure ... shows the difference between FEA and the standard EA. In the following, each stages will be introduced and analyzed

\subsubsection{Initialization}

The population $Pop$ is initialized until reaching the initialization size of $p_{min}$ individuals. Age of each individual is set at the adult age from the beginning for the first crossover stage. The individuals in the population are initialized using two different methods:
\begin{itemize}
	\item The Interval Graph method: The region is divided into multiple of trapeziums based on the vertices of the obstacles. By that, the Interval Graph is built by drawing vertical lines from every vertices of the obstacles that intersect the boundaries of the region and other obstacles. The trapeziums are then created from these lines and intersection points. Two trapeziums are considered neighbor if they share an edge. The partitioning is illustrated by Figure .... The individual is initialized by: starting from the source trapezium (the one contains the source point), choosing a random neighbor trapezium until it reaches the destination trapezium. Then, from each shared edge of the selected trapeziums, a random point is generated. At last, the individual is created by connecting these random points and normalized by partitioning the path into $\Delta s$ line segments. The source point and the destination point is set automatically as the first and the last point of the individual. Figure ... illustrates the process of initializing an individual using the Interval Graph method.
	\item The Randomization method: The region is divided into $k$ equal area, that forms a simple Interval Graph. The individual is then generated the same way as the first method.
\end{itemize}
It can be seen that the initialized individuals may cross through the obstacles area in some cases because the initialization method can only generate forward paths and unable to generate backward paths. The individuals in these case is basically invalid and should be removed in common sense. However, to enlarge the search space, we keep these individuals in the population for materializing the very first crossover stages and expect their good attributes can be passed to the successors. The fitness values of these individuals will be set at a great number or infinity to reduce their impaction and prevent them from downgrading the quality of the population.

\subsubsection{Family Pairing}

Families are created by pairing single individuals randomly until it satisfies the pairing rate $R_{pair}$. Two random individuals will be chosen from $Single$, get paired and moved to $Family$ set. However, there is one constraint for choosing individuals that they must be over the adult age to get paired.

\subsubsection{Crossover}

In the previous works, all of the algorithms have a drawback that the intruder is not allowed to move backward and can only move forward, which makes it not applicable in realistic scenarios. Besides, the obstacle model in O-based-MEP often requires the intruder to move backward in many cases. To overcome this, in FEA, the Improved-Leaning Crossover is used for crossover stage. The idea of the Improved-Leaning Crossover is simple: choose on each parent individual a random point and then connect these two point using a suitable method.
To process this operation, firstly, only the paired individuals which are placed in $Family$ can perform the crossover stage with their paired mate. Each pair then processes the Improved-Leaning Crossover operator and their created children are placed in a temporary set $Ch1$. The set $Ch1$ is then being used as the input for the Mutation stage.


\subsubsection{Mutation}

From the children set $Ch1$, a number of individuals are selected to be mutated with a rate determined by the mutation rate $R_{muta}$. After the Mutation stage, the output individuals set $Ch2$ will be added to $Single$ as they are not adult as well as paired. In FEA, two mutation operators are used: the Mid-point Mutation and the Rotation Mutation. ... 

\subsubsection{Update}

In this stage, the age of each individual is calculated and its state will be updated correspondingly. Individuals with age surpass the death age $D$ will be marked as dead and removed from the population. If the dead individual is currently in a Family, its paired individual will be set free and moved to $Single$. However, the best individual of the population will be preserved and can not be removed even after surpass the death age. This condition is added to make sure that the best fitness value of the population is better after each generation.

\subsubsection{Selection}

Selection stage is used to control the size of the population and triggered only when the number of individuals in $Pop$ surpass the upper bound $P_{max}$. The population size will then be adjusted by a selection process. The individuals in the population will be ordered by their fitness values and the lowest $p_{min}$ individuals will be selected. The unselected individuals will be removed from the population. If an individual in $Family$ is selected but its paired individual is removed, the individual will be moved to $Single$.

\subsubsection{Family System based Evolution Algorithm}

Summarize the above stages, FEA algorithm is proposed as following steps:

\begin{enumerate}
	\item \textbf{Initialization}: $P_{min}$ individuals is initialized using the initialization methods and added to the $Single$ set of the population.
	\item \textbf{Family pairing}: Individuals having age over the adult age $A$ in $Single$ are randomly chosen, get paired and added to $Family$ until it reaches the pairing rate $R_{pair}$.
	\item \textbf{Crossover}: Each pair performs the crossover process and the created children is added to a temporary set $Ch1$
	\item \textbf{Mutation}: Individuals in $Ch1$ are mutated with the rate of $R_{muta}$ and the output individuals are added to $Single$ set.
	\item \textbf{Update}: The age of each individual is calculated and its state will be updated correspondingly.
	\item \textbf{Selection}: If the size of the population is over $p_{max}$ then the best $p_{min}$ individuals will be kept for the next generation.
	\item \textbf{Terminal Condition}: If the number of generations reaches a fixed value, the algorithm is terminated and the best individual is returned as the MEP. Whereas, the process comes back to step 2.
\end{enumerate}

\subsection{Algorithm Pseudo Code}

At this section, the pseudo codes for the proposed algorithm will be represented.

\begin{algorithm}[H]
	\SetAlgoLined
	\KwIn{
		\begin{itemize}
			\itemsep-0.2em
			\item The pairing rate $R_{pair}$ \\
			\item The population $Pop$ \\
			\item The Families set $Family$ \\
			\item The Single set $Single$ \\
	\end{itemize}}	
	\KwOut{\\The updated Families set $Family$\\}
	\Begin{
		$n_{family}$= $R_{pair}*Pop.size / 100$;\\
		\While{ $Family.size<n_{family}$ }
		{
			Randomly select two individuals in $Single$ with age over $A$; \\
			Pair the two individuals as a Family and add to $Family$; \\
		}
	}
	\caption{\textbf{Family Pairing}} 
	\label{alg.1}
\end{algorithm} 

In the algorithm for Crossover Operator and Mutation Operator, we use the definition of  the Delta-cut. the Delta-cut of a line segment $EF$ is a sequence of points $\{P_1,P_2,...,P_m\}$ where $EP_1=P_1P_2=P_2P_3=....=P_{m-1}P_m=\Delta_x$ and $P_mF \leq \Delta_x$.\\

\begin{algorithm}[H]
	\SetAlgoLined
	\KwIn{
		\begin{itemize}
			\itemsep-0.2em
			\item The father individual $(P_1^a,P_2^a,…,P_{n_a}^a)$ \\
			\item The mother individual $(P_1^b,P_2^b,…,P_{n_b}^b)$ \\
	\end{itemize}
	}
	\KwOut{
		\\The children $child1$ and $child2$\\}
	\Begin{
		Randomly generate: $k_1$ in range $ [1,n_a ]$ and $k_2$ in range $ [1,n_b ]$;\\
		Let $Y_a$ is the Delta-cut of the line segment  $ P_{k_1}^a P_{k_2}^b $;\\
		$child1=(P_1^a,P_2^a,\ldots,P_{k_1}^a,Y_a,P_{k_2}^b,\ldots,P_{n_b}^b)$;\\
		Let $Y_b$ is the Delta-cut of the line segment  $ p_{k_2}^b P_{k_1}^a $;\\
		$child2=(P_1^b,P_2^b,\ldots,P_{k_2}^b,Y_b,P_{k_1}^a,\ldots,P_{n_a}^a)$;\\
	}
	\caption{\textbf{Crossover Operator}} 
	\label{alg.2}
\end{algorithm} 

\begin{algorithm}[H]
	\SetAlgoLined
	\KwIn{
		\begin{itemize}
			\itemsep-0.2em
			\item The selected individual $(P_1,P_2,\ldots,P_n)$ \\
			\item The mutation parameter $ \Delta_m $ \\
		\end{itemize}
	}
	\KwOut{\\The mutated individual $muta$ \\}
	\Begin{
		Randomly generate: $k_1$ and $k_2$ in range $ [1,n ]$;\\
		Create a point $Z$ where:\\
		$Z.x=random(0,\Delta_m\times2+max(P_{k_1}.x,P_{k_2}.x)-min(P_{k_1}.x,P_{k_2}.x))-\Delta_m+min(P_{k_1}.x,P_{k_2}.x) $\\
		$Z.y=random(0,\Delta_m\times2+max(P_{k_1}.y,P_{k_2}.y)-min(P_{k_1}.y,P_{k_2}.y))-\Delta_m+min(P_{k_1}.y,P_{k_2}.y)$;\\
		Let $Y_1$ is the Delta-cut of the line segment  $ P_{k_1} Z $;\\
		Let $Y_2$ is the Delta-cut of the line segment  $ Z P_{k_2} $;\\
		Let $muta = (P_1,P_2,\ldots,P_{k_1},Y_1,Y_2,P_{k_2},\ldots,P_{n})$;\\
	}
	\caption{\textbf{Mid-point Mutation Operator}} 
	\label{alg.3}
\end{algorithm} 

\begin{algorithm}[h]
	\SetAlgoLined
	\KwIn{
		\begin{itemize}
			\itemsep-0.2em
			\item The selected individual $(P_1,P_2,\ldots,P_n)$ \\
			\item The mutation parameter $ \Delta_m $ \\
		\end{itemize}
	}
	\KwOut{\\The mutated individual $muta$ \\}
	\Begin{
		Randomly generate: $k_1$ and $k_2$ in range $ [2,n-1]$;\\
		(Let $r=random(0,\Delta_m)$ and $\alpha = random(0,2\pi)$;\\
		Translate the segment $P_{k_1}P_{k_2}$ for a distance $r\sin\alpha$ on x-coordinate and $r\cos\alpha$ on y-coordinate;\\
		Let the new position of segment $P_{k_1}P_{k_2}$ is $P'_{k_1}P'_{k_2}$
		Let $Y_1$ is the Delta-cut of the line segment  $ P_{k_1-1} P'_{k_1} $;\\
		Let $Y_2$ is the Delta-cut of the line segment  $ P'_{k_2} P_{k-2+1} $;\\
		$muta = (P_1,P_2,\ldots,P_{k_1-1},Y_1,P'_{k_1},\ldots,P'_{k_2},Y_2,P_{k_2+1},\ldots,P_{n})$;\\
	}
	\caption{\textbf{Rotation Mutation Operator}} 
	\label{alg.4}
\end{algorithm} 


\section{Experimental results}
In this section, first,  After that, the experimental results are presented and includes of two following parts: 
\begin{itemize}
	\item \textbf{Experiment Setting}: building a dataset that contains various scenarios for the randomization experiment. A number of parameters setting as well as system environment for the execution are also established.
	\item \textbf{Parameters trial scenarios}: experimenting different values of important parameters to explore how it affects the performance of the proposed algorithm.
	\item \textbf{Topologies trial scenarios}: using various topologies to delve into the effectiveness of our algorithms in different network scenarios. In order to prove the effectiveness of FEA, the proposed algorithm is also compared to exist approaches to the problem.
\end{itemize} 

\subsection{Experiment Setting}
\subsubsection{Dataset}

In order to assess the performance of FGA under different scenarios, the algorithm is tested with a randomized dataset. The dataset must contain a number of randomly generated obstacles and sensors without violating the constraints of the OMEP model that no sensor is deployed inside the obstacle area. In this paper, we propose a method to randomly generate topologies based on three changeable values: the number of obstacles $n_o$, the total area of obstacles $S_o$ and the number of sensors $n_s$. The method perform following steps: 

% , we propose a randomized algorithm to create $W \times H$ sensoring fields with a certain number of obstacles covering a known percentage of area and a certain number of uniformly distributed censors that do not lie in any constructed obstacle.

\begin{enumerate}
	\item Distribute the total area of obstacles $S_o$, by randomly generate $n_o$ numbers with the fixed sum of $S_o$.
	\item Construct each obstacle with fixed area by:
	\begin{itemize}
		\item Randomly generate the number of vertices of the obstacle $V_o$ . In here, the number of vertices is bounded in range [3,6] for the purpose of simple simulation and computation. After that, generate the angles of the obstacle 
		\item Generate the angles of the obstacle. A polygon with $V_o$ vertices has angles sum up to $(V_o - 2).\pi$. Hence, this step can be done by randomly generate $V_o$ numbers with the fixed sum of $(V_o - 2).\pi$.
		\item Generate a random ray $Ox$ as a base.
		\item Generate each side of the obstacle one by one, with $O_x$ as the base. The angles between two consequence sides and the angle between the first side and $O_x$ are based on the output of the last step. The length of each side is randomized between [0,1]. The last side must intersect with ray $O_x$, otherwise, the process is repeated.
		\item Resize the obstacle to the preferred area by a homothetic transform with a condition that the obstacle can be placed within the field $\Omega$.
	\end{itemize}
	\item Locate these obstacles such that they do not overlay with each other and do not exceed the sensors field by repeatedly randomization until valid.
	\item Randomly deploy sensors in the field $\Omega$ without any sensors placed in any generated obstacles.
\end{enumerate}



%Firstly, we will construct a randomized algorithm to obtain the set of obstacles. The obstacles are convex polygons cover the area that block the moving path of object and also sensoring signal from sensors.
%\begin{enumerate}
%	\item The first step is to randomly distribute the total obstacle area into a certain number of obstacles. The method is presented as follow:
%	\begin{itemize}
%		\item Firstly, to obtain $n$ numbers with the sum of $N$, we chooses $n - 1$ uniformly distributed random numbers in the interval $[0; N]$, along with $0$ and $N$ we have an array of $n + 1$ numbers lie in the interval $[0; N]$.
%		\item Sort the obtained array in descending order.
%		\item If we continuously do a subtraction between 2 consecutive terms of the previous array, we will obtain $n$ uniformly distributed random numbers that have the sum of $N$.
%	\end{itemize}
%	As we have the random area of each obstacle, now we will construct each separately.
%	\item Consider a random known-area obstacle, firstly, we can see that in order to construct it, we can ignore the area constraint since this constrain can easily be satisfied through a homothetic transformation, and secondly, we can bounds the sides in an finite interval from 0 for the same reason. With these 2 observations, we can start constructing the polygon as follow:
%	\begin{itemize}
%		\item Choose the number of vertices of the polygon ($V$), this can be done through any randomised function. In our article, we use a quadratic equation to transform an uniform distribution into one varies from 3 to 6 with the peak at 4. Hence, the chance of the polygon to have 4 vertices is greatest.
%		\item Obtain the array of angle. It is known that a polygon with $V$ vertices has angles sum up to $(V - 2).\pi$. Hence, the problem can be done with the same method as described in 1.
%		\item Construct $V - 2$ sides of the polygon one be one, each side make a known angle with the previous side, has a random length between 0 and 1. With the exception of the first side can lie in an arbitrary angle.
%		\item The last 2 sides are constructed by the rays present the angle made by them and their neighbours. If these 2 rays intersect, the algorithm expires, return a random polygon with arbitrary area, else the previous step will be repeat until a polygon is created.
%		\item The last step of the method is a homothety to resize the polygon with the known original area. A check occurs to make sure the circumscribe rectangle of the polygon can lies inside the field, otherwise these step will repeat until a satisfied polygon is constructed.
%	\end{itemize}
%	After finishing these steps, we obtain a set of obstacles having certain areas and can lie inside the sensoring field, the last step is only to locate these obstacles such that they do not overlay with each other and do not exceed the sensoring field.
%	\item The last step of the algorithm is to randomly locate the position of the obstacles. This can again be done through trial and error method, the obstacles are repeatedly put inside the field until they are all satisfied, the obstacles construction can be redone if necessary.
%\end{enumerate}
%
%After these step we have successfully construct a random set of obstacle. It is also noticeable that this constructing algorithm can be manipulated easily to constrain the properties of the obstacles for convenience. For example, we can bound the area of obstacles such that no obstacle is too small or too big, or we can bound the angles of the obstacle to obtain a result more preferable to our application purposes.
%
%The final step of creating field data is to randomly deploy sensors inside the area. The sensors are put continuously in the field following certain distribution rules until it lies outside every obstacle exist.

To create various scenarios for experimental purpose, the three changeable values are set as following:
\begin{itemize}
	\item The number of obstacles $n_o$ : 3 , 5.
	\item The total area of obstacles $S_o$: 15\%, 30\% , 50\% of the field $\Omega$.
	\item The number of deployed sensors $n_s$ : 30, 60, 90.
\end{itemize}
With each combination of these values, we randomly generate 5 topologies using the above algorithm. In total, our dataset includes of 90 different topologies for experiment.

\subsubsection{Parameters}
\begin{itemize}
	\item The dimension of sensor field: $ W $ = 500, $ H $ =500
	\item Source point $ B $: (0, 150) 
	\item Destination point $ E $: (500, 350)	
\end{itemize}
All simulations were experimented on a machine with Intel® Core™ i7-3230M 2.60 GHz with 8 GB of RAM under Windows 10 64 bit using Java language.
\subsection{Parameters trials}
The performance of the algorithms can be effected when major parameters are set in different way. In this section, various versions of the proposed algorithms with different parameters setting will be tested to find out the best version of parameters set.

\subsection{Topologies trials}
\begin{landscape}
	\include{ResultsType6EUC_Exist_Alg1}
\end{landscape}
\end{document}