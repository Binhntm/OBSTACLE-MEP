\documentclass[review]{elsarticle}

\usepackage{lineno,hyperref}
\modulolinenumbers[5]

\journal{Journal of \LaTeX\ Templates}

%%%%%%%%%%%%%%%%%%%%%%%
%% Elsevier bibliography styles
%%%%%%%%%%%%%%%%%%%%%%%
%% To change the style, put a % in front of the second line of the current style and
%% remove the % from the second line of the style you would like to use.
%%%%%%%%%%%%%%%%%%%%%%%

%% Numbered
%\bibliographystyle{model1-num-names}

%% Numbered without titles
%\bibliographystyle{model1a-num-names}

%% Harvard
%\bibliographystyle{model2-names.bst}\biboptions{authoryear}

%% Vancouver numbered
%\usepackage{numcompress}\bibliographystyle{model3-num-names}

%% Vancouver name/year
%\usepackage{numcompress}\bibliographystyle{model4-names}\biboptions{authoryear}

%% APA style
%\bibliographystyle{model5-names}\biboptions{authoryear}

%% AMA style
%\usepackage{numcompress}\bibliographystyle{model6-num-names}

%% `Elsevier LaTeX' style
\bibliographystyle{elsarticle-num}
%%%%%%%%%%%%%%%%%%%%%%%

\begin{document}

\begin{frontmatter}

\title{Elsevier \LaTeX\ template\tnoteref{mytitlenote}}
\tnotetext[mytitlenote]{Fully documented templates are available in the elsarticle package on \href{http://www.ctan.org/tex-archive/macros/latex/contrib/elsarticle}{CTAN}.}

%% Group authors per affiliation:
\author{Elsevier\fnref{myfootnote}}
\address{Radarweg 29, Amsterdam}
\fntext[myfootnote]{Since 1880.}

%% or include affiliations in footnotes:
\author[mymainaddress,mysecondaryaddress]{Elsevier Inc}
\ead[url]{www.elsevier.com}

\author[mysecondaryaddress]{Global Customer Service\corref{mycorrespondingauthor}}
\cortext[mycorrespondingauthor]{Corresponding author}
\ead{support@elsevier.com}

\address[mymainaddress]{1600 John F Kennedy Boulevard, Philadelphia}
\address[mysecondaryaddress]{360 Park Avenue South, New York}

\begin{abstract}
This template helps you to create a properly formatted \LaTeX\ manuscript.
\end{abstract}

\begin{keyword}
\texttt{elsarticle.cls}\sep \LaTeX\sep Elsevier \sep template
\MSC[2010] 00-01\sep  99-00
\end{keyword}

\end{frontmatter}

\linenumbers

\section{The Elsevier article class}

\paragraph{Installation} If the document class \emph{elsarticle} is not available on your computer, you can download and install the system package \emph{texlive-publishers} (Linux) or install the \LaTeX\ package \emph{elsarticle} using the package manager of your \TeX\ installation, which is typically \TeX\ Live or Mik\TeX.

\paragraph{Usage} Once the package is properly installed, you can use the document class \emph{elsarticle} to create a manuscript. Please make sure that your manuscript follows the guidelines in the Guide for Authors of the relevant journal. It is not necessary to typeset your manuscript in exactly the same way as an article, unless you are submitting to a camera-ready copy (CRC) journal.

\paragraph{Functionality} The Elsevier article class is based on the standard article class and supports almost all of the functionality of that class. In addition, it features commands and options to format the
\begin{itemize}
\item document style
\item baselineskip
\item front matter
\item keywords and MSC codes
\item theorems, definitions and proofs
\item lables of enumerations
\item citation style and labeling.
\end{itemize}

\section{Front matter}

The author names and affiliations could be formatted in two ways:
\begin{enumerate}[(1)]
\item Group the authors per affiliation.
\item Use footnotes to indicate the affiliations.
\end{enumerate}
See the front matter of this document for examples. You are recommended to conform your choice to the journal you are submitting to.

\section{Bibliography styles}

There are various bibliography styles available. You can select the style of your choice in the preamble of this document. These styles are Elsevier styles based on standard styles like Harvard and Vancouver. Please use Bib\TeX\ to generate your bibliography and include DOIs whenever available.

Here are two sample references: \cite{Feynman1963118,Dirac1953888}.

\section{Proposed Algorithm}

%Emphasize the difference between standard and family Genetic Algorithm
%Describe the motivation of using GA and the motivation to modify the standard GA as we perform 

Without loss of generality, we can assume that the mobile object always moves at its greatest speed as the sensors system is stable, the mobile object should move as fast as possible to minimise total time of appearance in the sensor field. Hence, we can calculate the total exposure through the differentiate of not only time but also distance. %insert Appendix above

Consider a path $P$, since it is impossible to calculate the exact value, and acknowledging a proper approximation of the total exposure is enough for application purposes, we will divide the path by several points into small equidistant parts that has approximately equal exposure. As a result, the total value can evaluate through the sum of the exposure in the small parts, which will later be called as $S(P)$. And we can solve the minimal exposure path problem by finding the smallest $S(P)$ possible. Furthermore, since the parts are extremely small compared to the whole path. It is suitable to substitute the distance between two consecutive points on the path by their displacement ($d = \sqrt{dx ^ 2 + dy ^ 2}$)

In this section, we will introduce a Family-based Genetic Algorithm to effectively tackle this problem.

Due to the chaotic situation of the system of various sensors and obstacles, this problem is exceptionally complicated and contains numerous local minima, the Genetic Algorithm is a suitable choice because the diversity of the population and the stability of complexity may improve the solution over time, avoid being trapped in local minima and get preferable result in acceptable time.

From experimental result, we can see that (see Simulation results below), despite its known power and effectiveness, the original Genetic Algorithm seems not to be able to solve the minimal exposure with obstacle problem effectively. It can come from the fact that the individuals of the population can converge comprehensively leads to its falling into a local minima with very low probability of achieving better result.

Therefore, our proposed algorithm will try to find the proper paths while minimising the selection effect on the diversity of the community. Paths are constantly removed despite the total exposure along them, and the selection process only takes place when an over-population occurs. However, the algorithm also constantly stores good paths even after their elimination to archive diverse genes for future generation.

\subsection{Basic definitions}

Before getting further into details about our proposed algorithm, it is essential to explain the basic definitions throughout the performing of the algorithm.

\subsubsection{Individual presentation}

Each individual ($I$) stands for a path from the left edge to the right edge (see figure below). An individual is modelled with the list of equidisplacement points on that paths (the $n^{th}$ point on $I$ is called $I_n$). An individual has 2 variance states, age and fitness, with 2 static variables, adult age and death age.
%insert figure below
\begin{enumerate}
	\item The age of an individual ($t_I$) at a certain point is the number of periods has passed from its birth.
	\item The fitness of an individual ($f_I$) is the total exposure of the points in its model. From the statement above, our goal is to create an individual with the lowest fitness possible.
	\item The adult age ($t_A$) is the age at which a certain individual is ready for crossover process.
	\item The death age ($t_D$) is the age at which an individual is considered to be dead and removed from the population.
\end{enumerate}

\subsubsection{Population}

A population ($P$) is a set of individual we perform our algorithm on. It is expected that the lowest fitness of $P$ ($f(P)$) is constantly decreased each time the algorithm is performed, while the diversity, which is the number of non-similar individuals exist in the population, does not dramatically fall, in order to prevent the algorithm from falling into local minima. A population is consisted of 2 subsets (see figure below):
\begin{enumerate}
	\item Family ($F$) is the set of alive individuals that are partitioned into pairs and will constantly perform the crossover with its mate. The monogamy constraint increase the stability of population genes compared to random crossover situation, which will maximise the probability of getting preferable individuals and minimise the rate of producing terrible ones (details explained below).
	\item Singled ($S$) is the set of dead or alive individuals that are not paired. As a result, they will not perform crossover and their genes are not transmitted to other individual. Taking into consideration that there are also dead individuals in the Singled set, this comes from the fact that the Selection operation does not remove the individual from the Singled set and only marked it as dead.
	
\end{enumerate}
Moreover, a population contains some variables that will help the algorithm acknowledging its flow, in which there are three static variables and one variance variable:
\begin{enumerate}
	\item Current population ($p$) is the number of alive individuals currently appear in the population, this variable is the centre that control every activities of the algorithm.
	\item Min population ($p_{min}$) is the infimum of the current population. The current population reaches its minimum when the population is initialised at the beginning of the algorithm or each time the Selection process is triggered. Moreover, each time after the Selection occurs, if the Singled set has more elements than $p_{min}$, this set is also selected so that only $p_{min}$ individuals left in the Singled.
	\item Max population ($p_{max}$) is the supremum of the current population. Each time the current population exceeds this amount, a selection process occurs to remove the individuals with the highest fitness so that the population reduced to the $p_{min}$.
	\item Pairing rate ($R_{cross}$) is the lower limit of the ratio of the number of individuals which are paired into families. If the current ratio is lower than this infimum, the pairing process will occur.
	\item Mutation rate ($R_{mutation}$) is the rate of the individuals that are mutated each cycle of the algorithm. The mutation process helps diverse the population in order to avoid local minima and achieve preferable results. 
\end{enumerate}

\subsection{The proposed Family-based Genetic Algorithm}

The proposed algorithm is based on the standard Genetic Algorithm with significant modification on the operation of crossover process in order to avoid falling into local minima and maximising the efficiency of the algorithm.

A population is initially instantiated with $p_{min}$ random individuals, for convenience, all the original individuals' ages are set to $t_A$ so that the pairing and crossover processes can occur right from the beginning of the algorithm.

In each period of the algorithm, the processes are triggered consecutively in the  order: Pairing, Crossover, Mutation, Selection and Time update. The Pairing operation will put $p.R_{cross}$ random individuals together into families. These families will constantly create new individuals through crossover process. Then, the mutation

\subsubsection{Population initialisation}

\subsubsection{Pairing}

Each time the pairing process is called, it will remove several individuals in both the Single and the Death, they are putting into pairs and added to the Family. 

\subsubsection{Crossover}

\subsubsection{Mutation}

\section{Simulation results}

\subsection{Data generation}

Our experiments are on random field with uniform distribution of sensors and obstacles. The information of sensor field will be achieved through a random algorithm

\section*{References}

\bibliography{mybibfile}

\end{document}